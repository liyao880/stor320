% Options for packages loaded elsewhere
\PassOptionsToPackage{unicode}{hyperref}
\PassOptionsToPackage{hyphens}{url}
%
\documentclass[
]{article}
\usepackage{lmodern}
\usepackage{amssymb,amsmath}
\usepackage{ifxetex,ifluatex}
\ifnum 0\ifxetex 1\fi\ifluatex 1\fi=0 % if pdftex
  \usepackage[T1]{fontenc}
  \usepackage[utf8]{inputenc}
  \usepackage{textcomp} % provide euro and other symbols
\else % if luatex or xetex
  \usepackage{unicode-math}
  \defaultfontfeatures{Scale=MatchLowercase}
  \defaultfontfeatures[\rmfamily]{Ligatures=TeX,Scale=1}
\fi
% Use upquote if available, for straight quotes in verbatim environments
\IfFileExists{upquote.sty}{\usepackage{upquote}}{}
\IfFileExists{microtype.sty}{% use microtype if available
  \usepackage[]{microtype}
  \UseMicrotypeSet[protrusion]{basicmath} % disable protrusion for tt fonts
}{}
\makeatletter
\@ifundefined{KOMAClassName}{% if non-KOMA class
  \IfFileExists{parskip.sty}{%
    \usepackage{parskip}
  }{% else
    \setlength{\parindent}{0pt}
    \setlength{\parskip}{6pt plus 2pt minus 1pt}}
}{% if KOMA class
  \KOMAoptions{parskip=half}}
\makeatother
\usepackage{xcolor}
\IfFileExists{xurl.sty}{\usepackage{xurl}}{} % add URL line breaks if available
\IfFileExists{bookmark.sty}{\usepackage{bookmark}}{\usepackage{hyperref}}
\hypersetup{
  pdftitle={Final Presentation},
  hidelinks,
  pdfcreator={LaTeX via pandoc}}
\urlstyle{same} % disable monospaced font for URLs
\usepackage[margin=2]{geometry}
\usepackage{graphicx,grffile}
\makeatletter
\def\maxwidth{\ifdim\Gin@nat@width>\linewidth\linewidth\else\Gin@nat@width\fi}
\def\maxheight{\ifdim\Gin@nat@height>\textheight\textheight\else\Gin@nat@height\fi}
\makeatother
% Scale images if necessary, so that they will not overflow the page
% margins by default, and it is still possible to overwrite the defaults
% using explicit options in \includegraphics[width, height, ...]{}
\setkeys{Gin}{width=\maxwidth,height=\maxheight,keepaspectratio}
% Set default figure placement to htbp
\makeatletter
\def\fps@figure{htbp}
\makeatother
\setlength{\emergencystretch}{3em} % prevent overfull lines
\providecommand{\tightlist}{%
  \setlength{\itemsep}{0pt}\setlength{\parskip}{0pt}}
\setcounter{secnumdepth}{-\maxdimen} % remove section numbering

\title{\textbf{Final Presentation}}
\author{}
\date{\vspace{-2.5em}}

\begin{document}
\maketitle

{
\setcounter{tocdepth}{2}
\tableofcontents
}
\hypertarget{purpose}{%
\section{\texorpdfstring{\textbf{Purpose}}{Purpose}}\label{purpose}}

The purpose of the final presentation is to verbally communicate your
group's process to answer a question using data. A slideshow with
appropriate visuals and text should be utilized. This is your group's
opportunity to enthusiastically share your findings with your
classmates.

\hypertarget{requirements}{%
\section{\texorpdfstring{\textbf{Requirements}}{Requirements}}\label{requirements}}

In 5 to 7 minutes, you will focus your presentation on a single
question. Ideally, this should be 1 of the 2 questions you will write
about in your final paper. A slideshow should be used to organize the
information. First, you should creatively introduce the question and
orally defend why your group found this particular question interesting.
Next, you should briefly discuss the data your group used to answer this
question. The original source of the data should be given. The source is
not the website where you found the data, but who is responsible for
gathering the original data. Figures and tables can be used to highlight
variables of interest. Last, you should explain the statistical methods
you employed to answer the question and the results you discovered. I
recommend using multiple modeling techniques.

During the presentation, I will assess how well the question is
explained, how well the methods are explained, and how well the content
is organized on a scale from 0 to 3. The quality of slides will also be
scored on a scale from 0 to 3. Slides should not be cluttered, too
wordy, unreadable, or plain. A minimum of 4 visuals is required. Eye
contact and body language will also be assessed on a scale from 0 to 3.
The Orator should not be reading the slides. This is a short
presentation that should be well rehearsed.

The Orator is solely responsible for giving the presentation. All
members should help develop the slides, organize the information, and
provide an audience for the Orator to practice. Everyone should
proofread the slides since grammatical and spelling errors will result
in a loss of points. After 7 minutes, I will abruptly cut off your
presentation.

The final presentations will take place during the last two lectures.
Time slots will be prioritized according to email and posted on a google
spreadsheet linked on the website. I highly recommend scheduling your
meeting as soon as possible to ensure you have a preferred spot. The
Deliverer is responsible for submitting the slides to Gradescope. The
slides are due \textbf{before presentation day}. For example, if the
group is scheduled to present on April 30, the slides should be
submitted by April 29 (11:55PM). Attendance of all members is required.
Each member that fails to attend will automatically lose five points
from their individual grade. \textbf{Groups present early will get extra
credits}: 5 credits for groups present on April 30, and 2 credits for
groups present on May 3.

\end{document}
